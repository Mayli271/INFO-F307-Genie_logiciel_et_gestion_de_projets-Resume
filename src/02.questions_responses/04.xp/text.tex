\subsection{XP}


\subsubsection{Pourquoi appliquer XP ?}
L’extreme programming vise à supporter des changements réguliers des besoins.\\
$\Rightarrow$ incertitude des besoins.\\
Le principe de l’extreme programming est de fixer le temps, le prix et la qualité, et le paramètre variable sera l’étendue des fonctionnalités. Si le temps manque, au lieu d’éliminer les tests ou faire du code de mauvaise qualité, on supprime des fonctionnalités.


\subsubsection{Quels sont les valeurs de XP ?}
\begin{description}
	\item [Communication] Il faut oser dire qu’il y a des problèmes.
	\item [Simplicité] Un code simple marche mieux.
	\item [Feedback] Illustre échanges entre le client et les développeurs.
	\item [Courage] Ne pas avoir peur de faire des choix.
	\item [Respect] Entre les membres de l’équipe et le client.
\end{description}


\subsubsection{Quel est la principale pratique de XP ?}


\subsubsection{Comment concevoir avec XP ?}


\subsubsection{Comment un projet de XP est effectuée ?}
