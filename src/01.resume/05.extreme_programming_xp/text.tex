\section{Extreme programming (XP)}



\subsection{Description de l'XP}
L’extreme programming vise à supporter des changements réguliers des besoins.\\
$\Rightarrow$ incertitude des besoins.\\
Le principe de l’extreme programming est de fixer le temps, le prix et la qualité, et le paramètre variable sera l’étendue des fonctionnalités. Si le temps manque, au lieu d’éliminer les tests ou faire du code de mauvaise qualité, on supprime des fonctionnalités.



\subsection{Valeurs basiques de l'XP}
\begin{description}
	\item [Communication] Il faut oser dire qu’il y a des problèmes.
	\item [Simplicité] Un code simple marche mieux.
	\item [Feedback] Illustre échanges entre le client et les développeurs.
	\item [Courage] Ne pas avoir peur de faire des choix.
	\item [Respect] Entre les membres de l’équipe et le client.
\end{description}



\subsection{Principes}
Il y a les \textbf{pratiques primaires} qui font réellement partie de l’XP et sont nécessaires et les \textbf{pratiques secondaires} sont facultatives mais intensifient l’expérience.



\subsubsection{Programmation par paire}
Un des développeurs code, et l’autre dicte et explique ce qu’il doit coder. Ils font des rotations.



\subsubsection{Histoire}
Les \textbf{histoires} sont des besoins. Ce sont les seuls documents de besoins de l’XP. Elles sont très petites ( +- une phrase, un titre et une estimation de durée). Elles doivent être simples et maintenues à jour.
Elles ne comprennent rien de technique.



\subsubsection{L’intégration continue}
Le projet peut être découpé en plusieurs tâches, chacune faite par un développeur (ou un groupe de dèv.). La technique du \textbf{divide and conquer}. Il faut constamment intégrer ces morceaux.



\subsubsection{Test-first programming}
On code d’abord par coder le test, puis ensuite la fonctionnalité. Si le test est difficile à écrire, c’est que la fonctionnalité est mal spécifiée.



\subsubsection{Conclusion}
Le processus est incrémental ( Le logiciel “grandit”) et itératif (les dèv. apprenent pendant le développement.).