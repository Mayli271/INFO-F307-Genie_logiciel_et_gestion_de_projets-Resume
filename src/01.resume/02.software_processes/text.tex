\section{Processus logiciel}



\subsection{Qu'est-ce qu'un processus logiciel ?}
Un projet logiciel est composé d'activités (Planning, design, test, ...) et les activités sont organisés en différentes phases.

Un processus logiciel
\begin{itemize}
    \item définit l'ordre et fréquence des phases ;
    \item spécifie les critères pour passer d'une phase à l'autre ;
    \item définie si un projet est livrable ou non.
\end{itemize}



\subsection{Phases d'un processus logiciel}
\begin{description}
    \item [Lancement (Inception)] Le produit logiciel y est conçu et défini.
    \item [Organisation (Planning)] Initialise l'emploi du temps, les ressources et le coût déterminé.
    \item [Besoin d'analyse (Requirements Analysis)] Structure et spécifie les besoins ("Quoi").
    \item [Conception (Design)] Structure et spécifie une solution ("Comment").
    \item [Mise en oeuvre (Implementation)] Construit une solution du logiciel.
    \item [Test (Testing)] Valide la solution en fonction des besoins.
    \item [Entretien (Maintenance)] Répare les défauts et adapte la solution aux nouveaux besoins.
\end{description}



\subsection{Modèles de processus logiciel}



\subsubsection{Waterfall}
Le processsus classique utilisé comme modèle est le développement logiciel étape-par-étape \textbf{waterfall}.\\
On analyse les besoins du client, on design l'application, on la code, on la lire et ensuite on la maintient.\\
On termine une étape avant de commencer la suivante.

C'est facile à mettre en place mais on ne recoit aucun retour du client. Donc si il y a une erreur dans la compréhension des besoins du clients ; on ne s'en rend compte que à la fin du processus et il est trop tard. On doit alors revenir à la première étape et tout refaire, le processsus devient donc long et peut être coûteux.



\subsubsection{Itératif et incrémental}
Le développement \textbf{itératif} est un waterfall avec un feedback du client entre chaque étape.\\
Le développement \textbf{incrémental} consiste à développer le produit fonctionnalité par fonctionnalité en livrant à chaque fois une version préliminaire du projet (Processus composé de mini-waterfall).\\
La \textbf{livraison incrémentale} est la même chose mais appliqué à la livraison.



\subsubsection{Prototyping}
Prototyping est le processus de création d'un modèle incomplet du futur programme logiciel qui peut être utilisé pour des tests, exploration ou pour valider une hypothèse.


\subsubsection{Unified Process}



\subsubsection{Open Source}



\subsection{Documentation}
