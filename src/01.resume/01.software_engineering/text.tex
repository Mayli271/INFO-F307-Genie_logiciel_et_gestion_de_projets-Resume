\section{Introduction}



\subsection{Qu'est-ce que le génie logiciel ?}
\textbf{Le génie logiciel} est le fait de développer des logiciels en suivant les meilleurs méthodes possibles.



\subsection{Pourquoi le génie logiciel est très important ?}
Par exemple la fusée Ariane 5 qui a explosé suite à un petit bug dans un programme. C'est encore aujourd'hui le bug informatique le plus chère de l'histoire.



\subsection{Pourquoi un projet logiciel échoue ?}
Les projets logiciels échouent pour une ou plusieurs raisons :
\begin{itemize}
    \item Hors budget ;
    \item Retard, le client n'en aura peut être plus besoin ;
    \item Ne correspond pas aux exigences du client ;
    \item Qualité inférieure qu'initialement prévu ;
    \item Performances ne répondent pas aux attentes ;
    \item Trop difficile à utiliser.
\end{itemize}



\subsection{Les quatre P}
\begin{description}
    \item [People] Il y a plein de groupes de personnes qui interviennent sur le projet.
    \item [Product] Le code, le produit compil, la doc, les tests, ...
    \item [Project] Le planning, les processus de développement, les méthodologies utilisées.
    \item [Process] Manière d'organiser les gens, la discipline, la structure.
\end{description}



\subsection{Principes du génie logiciel}
Les plus importants :
\begin{itemize}
   \item La qualité est de première importance.
   \item Un logiciel de haute qualité est possible.
   \item Donner le produit aux clients le plus rapidement possible pour avoir leurs avis.
   \item Utiliser un processus de développement adapté.
\end{itemize}



\subsection{Éthiques du génie logiciel}
Il y a une part d'éthique. On ne copie/colle pas des bouts de codes sur des contrats différents, éthiques entre collègues, etc ...
