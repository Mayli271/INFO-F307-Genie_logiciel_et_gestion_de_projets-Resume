\section{Méthodes agiles}


\subsection{Développements agiles}
Les deux gros désavantages du waterfall sont :
\begin{itemize}
    \item Besoins doivent être parfaitement connu dès le départ.
    \item Le client ne voit que le produit fini.
\end{itemize}
La méthode agile se concentre sur \textbf{les gens}. Ils préfèrent un \textbf{logiciel qui marche} plutôt que de la \textbf{documentation}. La \textbf{collaboration avec le client} est mise en avant. Il faut savoir \textbf{répondre à des changements de besoins}.



\subsection{Principes de l'agile}
Les grands principes agiles sont les suivants:
\begin{itemize}
	\item Donner au clients de morceau de logiciel le plus souvent possible.
	\item Accepter les changements de besoins. ( Il y a toujours un code freeze quelques jours avant une livraison, où on n’ajoute plus de fonctionnalités et on corrige les bugs )
	\item Les gens du marketing et les développeurs doivent bosser ensemble.
	\item Engager des gens compétents et motivés et leur faire confiance.
	\item La meilleure manière de communiquer est en face-à-face.
	\item Un logiciel fonctionnel est la principale mesure de progression.
	\item Le rythme de travail doit pouvoir être maintenu sans se crever.
	\item Une attention constante à l’excellence technique et à un bon design.
	\item La simplicité est d’or.
	\item Les meilleures architectures et designs émergent naturellement d’équipes auto-organisées.
	\item L’équipe réfléchit régulièrement à comment se rendre plus efficace et ajuste ses procédures de travail de manière appropriée.
\end{itemize}



\subsection{Le cycle de l'agile}
Cycle: dure entre 1 et 6 semaines.
\begin{itemize}
	\item On obtient les besoins du client pour l’itération.
	\item On refactorise le code pour que les besoins puisse être ajoutés.
	\item Nouvelles fonctionnalités et des test.
	\item Code nettoyé le plus possible, simplifié.
\end{itemize}
A la fin on a un logiciel qui marche et testé.



\subsection{Intégration entre les processus agiles et non-agiles}
Il faut faire un compromis sur l’intensité de l’application du processus. Tous les idéaux ne sont pas adaptés au projet en cours.